\documentclass[class=Report, crop=false]{standalone}

\begin{document}
\section{Complete Instructions}
\label{sec:complinstr}

\begin{minted}{bash}
  module load cmake/3.10.2
  module load Python/3.6.4
  module load GCC/7.4.0
  module load MPICH/3.2.1

  export STACK_SRC=`mktemp -d /tmp/stack_temp.XXXXXX`
  export PACKAGES_DIR=$HOME/moose-compilers

  cd $STACK_SRC
  curl -L -O http://ftp.mcs.anl.gov/pub/petsc/release-snapshots/petsc-3.11.4.tar.gz
  tar -xf petsc-3.11.4.tar.gz -C .

  cd $STACK_SRC/petsc-3.11.4

  ./configure \
  --prefix=$PACKAGES_DIR/petsc-3.11.4 \
  --with-debugging=0 \
  --with-ssl=0 \
  --with-pic=1 \
  --with-openmp=1 \
  --with-mpi=1 \
  --with-shared-libraries=1 \
  --with-cxx-dialect=C++11 \
  --with-fortran-bindings=0 \
  --with-sowing=0 \
  --download-hypre=1 \
  --download-fblaslapack=1 \
  --download-metis=1 \
  --download-ptscotch=1 \
  --download-parmetis=1 \
  --download-superlu_dist=1 \
  --download-scalapack=1 \
  --download-mumps=1 \
  --download-slepc=1 \
  PETSC_DIR=`pwd` PETSC_ARCH=linux-opt

  make PETSC_DIR=$STACK_SRC/petsc-3.11.4 PETSC_ARCH=linux-opt all
  make PETSC_DIR=$STACK_SRC/petsc-3.11.4 PETSC_ARCH=linux-opt install
  make PETSC_DIR=$PACKAGES_DIR/petsc-3.11.4 PETSC_ARCH="" test

  if [ -d "$STACK_SRC" ]; then rm -rf "$STACK_SRC"; fi

  mkdir ~/projects
  cd ~/projects
  git clone https://github.com/hugary1995/raccoon.git
  cd raccoon
  git checkout master
  git submodule update --init

  cd moose/scripts
  METHOD=opt ./update_and_rebuild_libmesh.sh --with-boost

  cd ../..
  make -j N
  ./run_test -j N
\end{minted}

\end{document}
